\documentclass[a4paper,12pt]{article}

% config
\usepackage[brazil]{babel}
\usepackage[T1]{fontenc}
\usepackage{times}
\usepackage{hyperref}
\usepackage{amssymb}
\usepackage[pdftex]{graphicx}
\usepackage{setspace}
\usepackage{indentfirst}
\usepackage{graphicx}
\usepackage{subcaption}
\usepackage{textfit}
\usepackage[natbib=true,sorting=none, firstinits=true]{biblatex}
\usepackage{url}
\usepackage{color}  
\usepackage{wrapfig}
\usepackage[top=30mm,bottom=20mm,left=30mm,right=20mm,twoside]{geometry}

\usepackage{array}
\usepackage{multirow}

\hypersetup{
        colorlinks=true,
        citecolor=black,
        filecolor=black,
        linkcolor=black,
        urlcolor=black
}

% indentation
\setlength{\parindent}{0pt} % no indent
\setlength{\parskip}{2.0ex plus 0.5ex minus 0.2ex} % paragraph spacing
\setlength{\skip\footins}{2cm} % spacing between footer and text
\setlength{\footnotesep}{0.5cm} % space between footer itens
\renewcommand{\baselinestretch}{1.5} % line spacing

\bibliography{references}

\title{}

\author{Francisco Morgani Fatore}

\begin{document}

\hrule
{\large\bf How to Read a Visualization Research Paper: Extracting the Essentials} 
\vspace{0.25cm}
{\large Laramee, Robert S.}
\hrule

\vspace{1.0cm}
{\bf Conceito/Contribui��es:}

Este paper ensina como organizar a leitura de artigos. Busca ajudar na tarefa de ler uma grande quantidade de arquivos por meio de um m�todo para extrair as informa��es mais importantes dos documentos.

\vspace{0.5cm}
{\bf Implementa��o:}

O autor define alguns focos que o leitor deve manter ao ler um artigo.

\vspace{0.5cm}
{\bf Trabalhos relacionados: }

Existem trabalhos similares aplicados a outras �reas, mas do conhecimento do autor este artigo � o �nico que aborda a �rea de visualiza��o. 

O autor tamb�m relata trabalhos de outros pontos de vista como de refeere e revisores, um how-to-cheat para escrita de artigos de visualiza��o e um sobre como evitar que seu artigo seja rejeitado em uma revis�o.

Este artigo faz parte de uma colet�ncia (The PhD in Visualization Starter Kit) que possui artigos relacionados � outras habilidades exigidas por um candidato � doutorado/mestrado.

\vspace{0.5cm}
{\bf Trabalhos Futuros:}

Keeping up com a explos�o da literatura � um problema em aberto. Fornecer t�cnicas e ferramentas para explorar essa dificuldade pode ser um trablaho interessante.


\vspace{0.5cm}
{\bf Uso em outras �reas:}

O artigo pode ajudar pesquisadores tamb�m de outras �reas.

\printbibliography

\end{document}

